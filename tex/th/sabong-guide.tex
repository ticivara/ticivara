\chapter{สบง}

ผ้าสบงคือผืนหนึ่งใน 3 ผืนของชุดไตรจีวร ที่พระภิกษุสามเณรใช้สำหรับนุ่ง
เรียกชื่อตามพระวินัยว่า \emph{``อันตรวาสก''} อ่านว่า
\emph{``อัน-ตะ-ระ-วา-สก หรือ อัน-ตะ-ระ-วา-สะ-กะ''}
มีลักษระเป้นสี่เหลี่ยมผืนผ้า เวลานุ่งให้ด้านบนปิดสะดือ
และด้านล่างปิดครึ่งแข้ง

\section{การเลือกผ้า}

สบง ปกติจะใช้ผ้าที่หนาเช่นผ้าซันฟอไรซ์
ที่มีผ้าเนื้อหนาละเอียดซับเหงื่อได้ดี
หรืออาจจะใช้ผ้าด้ายดิบหรือผ้าไหมก็ได้ขอให้มีเนื้อผ้าหนา

\section{การกะขนาดขนาดผ้า}

\emph{ด้านยาว:} สบงที่นิยมจะมีขันธ์ 5 ขันธ์ มีความยาวไม่ตำกว่า 230 ซม.
\emph{(เพื่อป้องกันผ้าแหวกเวลานุ่ง)} แต่ไม่ควรยาวเกิน 250 ซม.
เพราะจะเกินขนาดที่พระวินัยเจ้าบัญญัติ \emph{ด้านสูง:} ยืนตัวตรง
ใช้สายวัด วัดเหนือสะดือลงมาถึงครึ่งหน้าแข้ง
นั่นคือขนาดความสูงของสบงที่เหมาะ

\section{การคำนวณในการตัด}

จะเหมือนกันกับการคำนวณจีวร เพียงแต่ลดจำนวนขัณฑ์ในการคำนวนลงเท่านั้น

\section{การเย็บและย้อม}

ขั้นตอนจะเหมือนกับการเย็บย้อมจีวรทุกประการ

