\chapter{อนุวาต}

\section{การตัด}

ถ้ามีพื้นที่พอบนแผ่นผ้า ใช้ส่วนที่เหลือด้านบนเพื่อทำอนุวาต (ผ้าสาบ)ด้านยาว
และใช้ด้านขวางสำหรับด้านสูง หรือตัดผ้าตามแนวหน้าผ้าของแต่ละด้าน
ทางผ้าของอนุวาตต้องเป็นทางเดียวกันกับผ้าผืนหลัก
ไม่เช่นนั้นอาจจะเกิดปัญหาผ้าย่นเพราะการหดตัวต่างกันของทางผ้าเมื่อผ่านการซัก

\photo{../static/img/sanghati/figures/border-orientation-vertical.jpg}

พับจากขอบผ้า ประมาณ 0.5-1 ซม. รีดชอบผ้าด้วยการทับเบาๆ
ด้วยเตารีดเพื่อให้เกิดรอยง่ายต่อการกดรีดแรงขึ้น
เมื่อรีดขอบทางยาวทังสองข้างแล้วม้วนเก็บโดยให้รอยรีดอยู่ด้านนอก ผูกด้วยเศษผ้าเก็บให้เรียบร้อย
ภาพข้างล่างเป้นตัวอย่างอนุวาตที่รีดและม้วนเก็บพร้อมที่จะใช้งาน

\photo{../static/img/sanghati/photos/borders-rolled-up-w500.jpg}

หากความยาวของหน้าผ้าไม่พอกับความยาวอนุวาตทีต้องการ จำเป็นต้องต่อ

\photo{../static/img/sanghati/figures/border-orientation-horizontal.jpg}

\setlength{\nextPhotoWidth}{0.9\textwidth}

\photo{../static/img/sanghati/figures/border-cut-and-join.jpg}

\section{การเย็บ}

\photo{../static/img/borders/figures/trim-area.jpg}

\photo{../static/img/borders/photos/mark-trim-area-w500.jpg}

\photo{../static/img/borders/photos/mark-trim-area-side-w500.jpg}

