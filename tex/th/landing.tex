\chapter{ไตรจีวร}

คือผ้า 3 ผืนที่เป็นหนึ่งในบริขาร 8 อย่าง \emph{``อัฏฐบริขาร''} ที่พระภิกษุจำเป็นต้องมี
ประกอบด้วยผ้าสามผืน อันได้แก่ ผ้าสบง \emph{``อันตรวาสก''} ผ้าจีวร
\emph{``อุตตราสงค์''} และผ้าห่มซ้อน \emph{``สังฆาฏิ''}

ในสมัยต้นพุทธกาลผ้าที่นำมาทำเป็นไตรจีวรยังมีการใช้ผ้าทั้งผืน
หรือนำเก็บผ้ามาต่อกันอย่างไม่มีรูปแบบ
เมื่อมีการลักขโมยผ้าของพระบ่อยครั้งเนื่องจากผ้าในสมัยนั้นหายาก
พระพุทธองค์จึงโปรดให้พระอานนท์ออกแบบทำผ้าเป็นลายคันนาเพื่อทำลายราคาของผ้า
จึงเกิดเป็นจีวรขันธ์สืบทอดมาถึงทุกวันนี้

การตัดเย็บจีวรเป็นภูมิปัญญาจากวัด ซึ่งสืบทอดกันต่อๆ
มาโดยเฉพาะพระสายวัดป่าจะฝึกให้สามเณรตัดเย็บไตรจีวรของตนเองเพื่อใช้ในการบวช

การคำนวณที่ใช้ในเว็บนี้เป็นเพียงหนึ่งของวิธีที่หลากหลายในการตัดเย็บ
หากมีข้อสงสัยสามารถสอบถามได้ที่ ratanawanno@gmail.com

