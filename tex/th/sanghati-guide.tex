\chapter{ผ้าสังฆาฏิ}

\section{การเลือกผ้า}

สังฆาฎิ ผ้าที่เหมาะแต่การตัดสังฆาฏิควรเป็นผ้าบาง เช่นป่านอินเดีย ป่านมัสลิน ผ้าลีนิน
ป่านอินเดียจะเป็นผ้าที่บางมากตัดเป็นสังฆาฏิแล้วจะหนาเท่ากับจีวรที่ตัดด้วยผ้าป่านมัสลิน

\section{การกะขนาดขนาดผ้า}

สังฆาฏิจะวิธีการวัดและขีดเช่นเดียวกันกับจีวร
จะแตกต่างก็ตรงที่ต้องตัดเส้นตัดของทุกขันธ์ให้แยกชิ้นกัน
และต้องทำขันธ์แต่ละขันธ์ให้มีขันธ์ที่เป็นเงาซึ่งกันและกัน เพื่อจะได้เย็บเป็นสองชั้นได้ ดังนั้นสังฆาฎิ
จะมีผ้าทั้งหมด 18 ชิ้น ชิ้นที่ 1-9 การขีดผ้าจะเหมือนกับจีวรทุกประการ

\section{การขีดขันธ์ที่เป็นเงา}

\begin{enumerate}
\def\labelenumi{(\arabic{enumi})}
\item
  ตัดผ้าทุกขันธ์ให้เท่ากับขันธ์ 1-9
\item
  นำขันธ์แต่ละขันธ์มาประกบกันทีละขันธ์
\item
  จุดตามเส้นของชุดแรก
\item
  พลิกผ้าขึ้นมาขีดเพื่อให้ทั้งชิ้น เป็นเงาซึ่งกันและกัน หมายความว่ามีเส้นแต่ละเส้นตรงกัน
  แต่คนละด้าน
\item
  พับผ้าซ้อนกันไว้แยกเป็นแต่ละขันธ์ เพื่อป้องกันความสับสน
\end{enumerate}

\section{การเย็บ}

\begin{enumerate}
\def\labelenumi{(\arabic{enumi})}
\tightlist
\item
  เย็บขึ้นกระดูกกุสิ \emph{เส้นขาด} แต่ละเส้นยาวประมาณ 10 ซม.
  ที่ไม่ต้องเย็บตลอดแนวเพราะเราจะเย็บเส้นนี้ตอนประกบผ้าสองชิ้นเข้ากัน เส้นอัฑฒกุสิ
  \emph{เส้นขวาง} เย็บตลอดแนว เย็บเช่นนี้ทุกๆ ชิ้น เก็บไว้เป็นคู่เพื่อกันลืม
\end{enumerate}

\setlength{\nextPhotoWidth}{0.5\textwidth}

\photo{../static/img/sanghati/photos/th-sanghati-sewing01.jpg}

\photoCaption{ขึ้นกระดูกผ้าทุกชิ้น}

\begin{enumerate}
\def\labelenumi{(\arabic{enumi})}
\setcounter{enumi}{1}
\tightlist
\item
  เมื่อเสร็จการขึ้นกระดูกแล้ว เริ่มประกบเส้นอัฑฒกุสิ (เส้นขวาง) ของขันธ์ที 5
  หรือขันธ์กลางด้านบน จุดสังเกตุคือ แนวกระดูกเส้นขาดของขันธ์
  และขัณ์ที่เป็นเงาต้องตรงกันทั้งสองด้าน ชายผ้าไม่เสมอไม่เป็นไร
  เมื่อได้ตำแหน่งที่ดีแล้วใช้เข็มหมุดยึดไว้ แล้วเย็บให้ชิดแนวตะเข็บเดิม
  ให้แนวตะเข็บใหม่อยู่ทางด้านผ้าชิ้นใหญ่ แนวตะเข็บเดิมอยู่ริมผ้า
\end{enumerate}

\setlength{\nextPhotoWidth}{0.45\textwidth}

\photo{../static/img/sanghati/photos/th-sanghati-sewing02.jpg}

\photoCaption{นำผ้าสองด้านมาประกบกระดูกเขาหากัน}

\setlength{\nextPhotoWidth}{0.45\textwidth}

\photo{../static/img/sanghati/photos/th-sanghati-sewing03.jpg}

\photoCaption{พับด้านบนลง ให้แนวเส้นกุสิ และอัฑฒะกุสิเสมอกันนำผ้าสองด้านมาประกบกระดูกเขาหากัน}

\setlength{\nextPhotoWidth}{0.45\textwidth}

\photo{../static/img/sanghati/photos/th-sanghati-sewing03.jpg}

\photoCaption{พับผืนล่างลงลักษณะเดียวกัน ได้ตำแหน่งดีแล้วกลัดด้วยเข็มหมุด}

\setlength{\nextPhotoWidth}{0.45\textwidth}

\photo{../static/img/sanghati/photos/th-sanghati-sewing04.jpg}

\photoCaption{เย็บเส้นอัฑฒะกุสิ หลังกระดูก เมื่อสำเร็จคลี่ผ้าออกจะเห็น ช่องอัฑฒะกุสิเป็นคลอง}

\begin{enumerate}
\def\labelenumi{(\arabic{enumi})}
\setcounter{enumi}{2}
\item
  เย็บเส้นอัฑฒะกุสิ เส้นขาดด้านล่าง ในลักษณะเดียวกับ เส้นด้านบน
  เมื่อเสร็จคลี่ผ้าออกจะเห็นช่อง อัฑฒะกุสิเป็นเหมือนคลอง มณฑลทั้งบนและล่างเป็นฝั่ง
\item
  ประกบเย็บเส้นกุสิ (เส้นขาด) เข้าด้วยกันทั้งสองด้าน ในขันธ์นี้จะมีทั้งหมดสี่เส้น
  จุดสังเกตุคือใช้แนวเส้นอัฑฒะกุสิเป็นจุดศูนย์กลางในการเย็บ เวลาพับ
  เข้าหาตัวและเมื่อพับแล้วต้องถูกห่อ
\end{enumerate}

\photo{../static/img/sanghati/photos/th-sanghati-sewing06.jpg}

\photoCaption{ประกบเส้นกุสิเข้าด้วยกัน}

\begin{enumerate}
\def\labelenumi{(\arabic{enumi})}
\setcounter{enumi}{4}
\tightlist
\item
  เมื่อประกบแนวเส้นกุสิ \emph{เส้นขาด} ทั้งสองด้านเข้าด้วยกัน
  จะสังเกตุว่าผ้าจะห่อเป็นถุงเสมอ
\end{enumerate}

\photo{../static/img/sanghati/photos/th-sanghati-sewing07.jpg}

\photoCaption{ด้านที่จะเย็บจะห่อส่วนอื่นเสมอ}

\begin{enumerate}
\def\labelenumi{(\arabic{enumi})}
\setcounter{enumi}{5}
\tightlist
\item
  เริ่มเย็บห่างจากจุดพับประมาณ 2-4 ซม. ผ้าจะได้ไม่ย่น และ เมื่อเย็บเสร็จแล้วผ้าจะมีสองชั้น
  มีลักษณะเป็นถุง ช่องกุสิ และอัฑฒะกุสิจะเป็นคลอง
\end{enumerate}

\photo{../static/img/sanghati/photos/th-sanghati-sewing08.jpg}

\photoCaption{เริ่มเย็บห่างจุดพับประมาณ 2-4 ซม.}

\clearpage

\subsection{ขันธ์ที่ 4 และ 6}

\begin{enumerate}
\def\labelenumi{(\arabic{enumi})}
\tightlist
\item
  พลิกผ้าของขันธ์ที่ 4 ให้คลุมขันธ์ที่ 5 จะเห็นว่าผ้ามี 4 ชั้น
\end{enumerate}

\setlength{\nextPhotoWidth}{0.6\textwidth}

\photo{../static/img/sanghati/photos/th-sanghati-sewing09.jpg}

\photoCaption{ซ้อนผ้าขันธ์ที่ 4 ให้คลุมขันธ์ 5}

\begin{enumerate}
\def\labelenumi{(\arabic{enumi})}
\setcounter{enumi}{1}
\tightlist
\item
  เย็บตามแนวเส้นตัด ให้ห่างจากชายผ้าพแประมาณเพื่อไม่ให้พลาดในการเย็บ
  และเพื่อต่อผ้าทั้งสี่ชั้นเข้าด้วยกัน ตรวจให้แน่ใจว่าผ้าทุกชิ้นเย็บติดกันดี
\end{enumerate}

\setlength{\nextPhotoWidth}{0.6\textwidth}

\photo{../static/img/sanghati/photos/th-sanghati-sewing10.jpg}

\photoCaption{ซ้อนผ้าขันธ์ที่ 4 ให้คลุมขันธ์ 5}

\clearpage

\begin{enumerate}
\def\labelenumi{(\arabic{enumi})}
\setcounter{enumi}{2}
\tightlist
\item
  ขลิบให้ตะเข็บเล็กที่สุด
\end{enumerate}

\photo{../static/img/sanghati/photos/th-sanghati-sewing11.jpg}

\photoCaption{ขลิบตะเข็บให้เล็กประมาณ 2 มิลลิเมตร}

\begin{enumerate}
\def\labelenumi{(\arabic{enumi})}
\setcounter{enumi}{3}
\item
  คลี่ผ้ามาเพื่อประกบเย็บเส้นอัฑฒกุสิของขันธ์ และขันธ์เงา
\item
  ประกบเย็บเส้นกุสิ (เส้นขาด) ของขันธ์ และขันธ์เงา
\item
  เมื่อเสร็จแล้วผ้าจะเป็น 2 ชั้น และมีลักษณะเป็นถุงเหมือนขันธ์ที่ 5
\item
  เย็บขันธ์ที่ 6 ในลักษณะเดียวกันกับขันธ์ 4 เพียงแต่อยู่คนละด้านของขันธ์ที่ 5
\end{enumerate}

\subsection{ขันธ์ที่ 7-9}

มีการเย็บลักษณะการเย็บเหมือนกับกับ ขันธ์ที่ 4 และ 6 เพียงแต่ต่อขันธ์ให้ถูกต้องตามลำดับเท่านั้น

\emph{เมื่อเย็บเสร็จแล้วจะเห็นว่าผ้าสังฆาฏิเป็นผ้าสองชั้น แต่ไม่เห็นรอยเย็บ การใส่อนุวาต
ลูกดุม และรังดุม ก็จะเหมือนกันกับจีวรทุกประการ}

