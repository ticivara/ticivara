ขนาดผ้าที่ตัดเย็บเรียบร้อยแล้วอาจจะหดเมื่อย้อมหรือซักขึ้นอยู่กับคุณภาพของผ้าที่ใช้ตัด
ดังนั้นขนาดการตัดจึงต้องคำนวณค่าความหดตัวของผ้าเพิ่มเข้าไปด้วย
ในการคำนวณนี้ไม่ได้เพิ่มข้าความหดเข้าไปในขนาดกุสิ
แต่จะเพิ่มเผื่อไว้ในส่วน ``ค่าพับเย็บอนุวาต'' และ ``เผื่อเพื่อขลิบ''
เพื่อง่ายต่อการขีดและตัด

\emph{ค่าพับเย็บอนุวาต} จะมีค่าประมาณ 0.5- 1 เซ็นติเมตร
\emph{เผื่อเพื่อขลิบ} สำหรับผ้าหนาและไม่ต่อผ้าเยอะ เพื่อไว้ประมาณ 5-10
เซ็นติเมตร หากเป็นผ้าบางหรือต้องต่อผ้าหลายชิ้นเผื่อไว้ประมาณ 20
เซนต์เมตร

หน่วยการวัดในการคำนวณใช้เป็นเซ็นติเมตร ไม่ได้ปัดเศษทศนิยม
ท่านสามารถใช้ค่าความยาวที่ค่อยๆ เพิ่มขึ้นที่แสดงไว้ขอบๆ
ช่วยให้ไม่ต้องยกไม้บรรทัดหรือสายวัดบ่อยๆ

