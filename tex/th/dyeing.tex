\chapter{การย้อม}

การย้อมทำได้สองลักษณะคือ ย้อมด้วยสีธรรมชาติ เช่นแก่นขนุนเคี่ยวล้วน
(อ่านขั้นตอนการทำสีแก่นขนุน และการย้อมสีแก่นขนุน ได้จากภาคผนวกเรื่อง สีจากแก่นขนุน) และ
ย้อมด้วยสีวิทยาศาสตร์ ในที่นี้จะอธิบายการย้อมด้วยสีวิทยาศาสตร์
เพราะหาและทำได้ง่ายกว่าสีธรรมชาติ

\section{อุปกรณ์}

\begin{enumerate}
\def\labelenumi{(\arabic{enumi})}
\item
  เตาสำหรับต้มน้ำร้อน
\item
  ภาชนะที่ใช้ต้ม มีปากกว้าง เช่นกระทะ กาละมัง หรือหม้อมีปากกว้าง ขนาดประมาณ 20 ลิตร
\item
  ภาชนะตักน้ำมีด้ามยาว เช่นกะบวย ที่กรองน้ำ เพื่อป้องกันเศษสี หรือ สิ่งสกปรกจากน้ำ
\item
  ภาชนะสำหรับย้อม เช่นกาละมัง ในที่นี้ใช้รางไม้ขุด
\end{enumerate}

\setlength{\nextPhotoWidth}{0.5\textwidth}

\photo{../static/img/dyeing/photos/th-dyeing-equipment01.jpg}

\photoCaption{รางย้อมผ้าและอุปกรณ์ที่ใช้ในการย้อม}

\photo{../static/img/dyeing/photos/th-dyeing-equipment02.jpg}

\photoCaption{อุปกรณ์การต้มน้ำย้อมผ้า ที่พอหาได้}

\begin{enumerate}
\def\labelenumi{(\arabic{enumi})}
\setcounter{enumi}{4}
\tightlist
\item
  ถุงมือพลาสติก แบบหนา และถังน้ำเย็นสำหรับช่วยลดความร้อนเวลาขยี้
\end{enumerate}

\photo{../static/img/dyeing/photos/th-dyeing-groves.jpg}

\photoCaption{ถุงมือพลาสติก และถังน้ำเย็น}

\clearpage

\begin{enumerate}
\def\labelenumi{(\arabic{enumi})}
\setcounter{enumi}{5}
\tightlist
\item
  สี ประกอบด้วย สีแก่นขนุน ตราสิงห์โต 1 กล่อง สีเหลือง ตราสิงห์โต 1 กลอ่ง สีกรัก
  ตรากิเลน ครึ่งกล่อง
\end{enumerate}

\photo{../static/img/dyeing/photos/th-dyeing-portions.jpg}

\photoCaption{อัตราส่วนการผสมสี}

\section{วิธีการย้อม}

\begin{enumerate}
\def\labelenumi{(\arabic{enumi})}
\item
  ต้มน้ำให้เดือด
\item
  นำสี 3 สีข้างต้น ลงภาชน ผสมน้ำร้อนด้วยน้ำพอประมาณท่วมผ้า คนให้เข้ากัน
\end{enumerate}

\photo{../static/img/dyeing/photos/th-dyeing-mixing.jpg}

\photoCaption{ผสมน้ำที่ต้มเดือด}

\photo{../static/img/dyeing/photos/th-dyeing-stiring.jpg}

\photoCaption{คนให้สีเข้ากัน}

\clearpage

\begin{enumerate}
\def\labelenumi{(\arabic{enumi})}
\setcounter{enumi}{2}
\tightlist
\item
  นำผ้าทุกชิ้นที่ชักด้วยน้ำเปล่า และบิดให้เสด็ดน้ำแล้ว ลงพร้อมกัน เพื่อให้สีที่ได้เสมอกัน
\end{enumerate}

\photo{../static/img/dyeing/photos/th-dyeing-dyeing.jpg}

\photoCaption{นำผ้าลงพร้อมกันขยี้ให้เร็ว}

\begin{enumerate}
\def\labelenumi{(\arabic{enumi})}
\setcounter{enumi}{3}
\tightlist
\item
  ขยี้ให้ผ้ากินสีเสมอกัน ใข้เวลาขยี้ประมาณ 10-15 นาที
\end{enumerate}

\photo{../static/img/dyeing/photos/th-dyeing-squeezing.jpg}

\photoCaption{ขยี้ให้ผ้ากินสีเสมอกัน}

\clearpage

\begin{enumerate}
\def\labelenumi{(\arabic{enumi})}
\setcounter{enumi}{4}
\tightlist
\item
  เมื่อได้สีเป็นที่พอใจแล้วให้นำผ้าไปผึ่งในที่ร่ม
\end{enumerate}

\photo{../static/img/dyeing/photos/th-dyeing-drying.jpg}

\photoCaption{นำไปตากในที่ร่ม}

\begin{enumerate}
\def\labelenumi{(\arabic{enumi})}
\setcounter{enumi}{5}
\tightlist
\item
  เมื่อแห้ง ให้ชุบด้วยน้ำต้มแก่นขนุนเข้มข้น เพื่อให้ได้สีและกลิ่นของแก่นขนุน
\end{enumerate}

\emph{การย้อมแบบนี้ สีจะตกต้องรักษากลิ่นและสีให้เหมือนธรรมชาติด้วย
การวิธีซักผ้าด้วยน้ำแก่นขนุนแทนการซักด้วยผงซักฟอก}

