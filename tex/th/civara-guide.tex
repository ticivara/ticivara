\chapter{ผ้าจีวร}

ผ้าจีวรเป็นผ้าผืนหนึ่งในสามผืนของผ้าไตรจีวร เป้นผ้าที่พระสงฆ์ใช้ห่ม
เรียกชื่อตามพระวินัยว่า \emph{อุตราสงค์} อ่านว่า \emph{``อุ-ตะ-รา-สง
หรือ อุ-ตะ-รา-สัง''} ห่มให้เป้นปริมณฑล (งาม หรือเหมาะสม)
ให้ด้านบนปิดหลุมคอ ชายด้านล่างปิดครึ่งแข้ง

\section{การเลือกผ้า}

จีวร ผ้าที่เหมาะต่อการตัดจีวรควรเป็นผ้าไม่หนามาก เช่นผ้าป่านมัสลิน
ป่านสวิส ลีนิน หรือป่านอินเดียเป็นต้น ทั้งนี้ขึ้นอยู่กับความชอบของผู้ใช้
ในพระวินัยอนุญาตให้ใช้ผ้าทุกขนิดยกเว้นผ้ากำพล คือผ้าที่ทอด้วยผมมนุษย์

\photo{../static/img/civara/th-civara-parts.jpg}

\photoCaption{ส่วนต่างๆ ของจีวร}

\section{การกะขนาดขนาดผ้า}

\textbf{ด้านกว้าง:} เพื่อให้ไม่เกิดความสับสนจะเรียกว่า
\_``ด้านสูง''\_มีวิธีการกะขนาดจีวรให้เท่ากับตัวผู้ใส่จะมีวิธีกะขนาดดังนี้
ให้เจ้าของจีวรจับตลับเมตร สายวัด หรือเชือกให้ยกแขนขึ้นด้านบนให้สูงทีสุด
แล้ววัดขนาดจากตำแหน่งที่จับ ถึงตำแหน่งหนึ่งผ่ามือขึ้นมาครึ่งแข้ง
ขนาดที่ได้ได้จะนำมาคำนวนต่อไป

\photo{../static/img/civara/th-civara-measuring01.jpg}

\photoCaption{การวัดหาด้านสูงของผ้า}

\textbf{ด้านยาว:} ให้ผู้ต้องการจีวรยืนตัวตรงเท้าชิด
จะใช้สายวัดหรือเชือกทบครึ่ง คล้องคอของผู้ต้องการจีวร
ปล่อยให้ปลายเชือกทั้งสองด้านแตะพื้นพอดี
แลัววัดดูความยาวของเชือกนั่นคือขนาดความยาวของจีวรที่เหมาะกับตัว

\photo{../static/img/civara/th-civara-measuring02.jpg}

\photoCaption{การวัดหาความยาวของผ้า}

\section{การคำนวนผ้าเพื่อตัด}

ก่อนอื่นเราต้องดูลักษณะของผ้าที่มีว่าเป็นผ้าหน้ากว้างเท่าไร
ควรจะแบ่งการตัดออกเป็นกี่ขันธ์ โดยปกติแล้วถ้าผ้ากว้างน้อยกว่า 115 ซม
ถือว่าเป็นผ้าหน้าแคบ

\section{การหาคำนวณหาค่าผ้าหดในการย้อม}

จากนั้นจะคำนวณหาส่วนหดในการเย็บย้อม ซึ่งมีวิธีหาดังนี้ คือ

\begin{enumerate}
\def\labelenumi{(\arabic{enumi})}
\item
  ตัดผ้าแต่ละด้านเป็นริ้วยาว 110 เซ็นติเมตร กว้างประมาณ 2.5 เซ็นติเมตร
  ขีดหัวท้ายในระหว่างผ้าด้วยปากกาลูกลื่นที่ซักไม่ออก ให้มีระยะห่าง 100
  เซ็นติเมตร
\item
  ทำสัญญาลักษณ์ ว่าผ้าด้านไหนให้ชัดเจน
  เพราะผ้าส่วนกว้างกับส่วนยาวจะหดไม่เท่ากัน
\item
  นำผ้านั้นไปต้มในน้ำเดือด 15 นาที
  หรือจะให้ได้ค่าที่แน่นอนกว่าก็ใช้วิธีการย้อม
\item
  ผึ่งหรือ รีดผ้าให้แห้ง
\item
  วัดดูว่าผ้าหดไปกี่เซ็นติเมตร เมื่อเทียบกับค่าก่อนการทดลอง
  ค่าที่ได้ตรงนี้เรียกว่า ``ค่าผ้าหดในการย้อม'' มีหน่วยเป็นเปอร์เซ็นต์
  ซึ่งจะนำไปใช้ในการคำนวณเวลาคำนวนเพิ่มความยาวผ้าในการตัด
\end{enumerate}

\setlength{\nextPhotoWidth}{0.5\textwidth}

\photo{../static/img/civara/th-civara-shrink-test01.jpg}

\photoCaption{ตัวอย่างการตัดผ้าในการการทดลองหาค่าผ้าหด}

\clearpage

\section{การกะขนาดขนาดผ้า}

\subsection{สูตรการคำนวณ}

\begin{quote}
{[}ค่าผ้าหดในการย้อม = ความยาวของผ้าก่อนการทดลอง --
ความยาวผ้าหลังการทดลอง (หน่วยเป็นเปอร์เซ็นต์){]}
\end{quote}

\begin{quote}
ตัวอย่าง 1

ขีดเส้นจัตุรัสขนาด 100 ซม. เมื่อทำตามขั้นตอนด้านบนแล้ววัดได้ว่า

หน้าผ้าด้านยาวเหลือ 98 ซม. แสดงว่า ผ้าหดไป 2 เปอร์เซ็นต์

หน้าผ้าด้านกว้างเหลือ 96 ซม. แสดงว่า ผ้าหดไป 4 เปอร์เซ็นต์

ค่าที่ได้นี้จะนำไปคำนวณส่วนหดของผ้าทั้งหมด
\end{quote}

\section{การคำนวนหาค่าผ้าหดในการเย็บ}

ในการเย็บหากเป็นผ้าชิ้นเดียว เราจะพับตรงเส้นเพื่อเย็บให้เป็นตะเข็บ
มีศัพย์เฉพาะเรียกว่า ``กระดูก'' (หากเป็นผ้าสองชิ้น
เมื่อเข้าผ้าก็จะเป็นกระดูก)
ต่อจากนั้นถึงเย็บรอบที่สองเพื่อทำการล้มกระดูก
เพื่อให้ผ้าที่พับทรือต่อแข็งแรง ในการพับเย็บจะเสียความยาวผ้าไปประมาณ 0.5
ซม. ส่วนในการต่อผ้าจะเสียความยาวผ้าไปประมาณ 0.7 ซม.

ผ้าจะเสียความยาวไปอีกในการพับเพื่อเย็บขอบผ้า ``อนุวาต''
จะเสียไปประมาณด้านละ 0.5- 1 ซม.

ค่าของการเย็บกระดูก และ ค่าพับเพื่อเย็บเข้าอนุวาต
จะนำไปใช้ในการคำนวณหาค่าของ ``ผ้าที่หดในการเย็บ'' มีขนาดเท่าไร
สรุปเป็นสูตร์ได้ดังนี้คือ

\begin{quote}
\begin{enumerate}
\def\labelenumi{(\arabic{enumi})}
\item
  จำนวนของกุสิ กุสิคือช่องรอยต่อระหว่างขันธ์ ทางด้านยาว
  ดังนั้นกุสิจะมีน้อยกว่าจำนวนขันธ์ อยู่ 1 เสมอ ตัวอย่าง หากสบงมี 5
  ขันธ์ ก็จะมีกุสิ อยู่ 4 และ จีวร 9 ขันธ์ ก็จะมีกุสิอยู่ 8
\item
  ขนาดจีวรที่ต้องการ คือขนาดที่เราได้วัด ดังแสดงในวิธีกะผ้า
\item
  จำนวนขันธ์ คือ จำนวนของตอนของผ้าที่เราต้องการ
  ตามที่นิยมทำทีวัดบึงแสนสุข และวัดทั่วๆ ไป สบงจะมีจำนวน 5 ขันธ์
  จีวรและสังฆาฎินิยม 9 จะมากว่านี้ก็ได้ แต่ต้องเป็นจำนวนคี่
\end{enumerate}
\end{quote}

\subsection{วิธีคำนวณ}

ซึ่งสามารถแทนค่าต่างๆ ได้ดังนี้

ค่าผ้าหดในการเย็บ-ย้อม แยกได้ดังนี้ คือ

\begin{enumerate}
\def\labelenumi{(\arabic{enumi})}
\item
  ค่าพับอนุวาตทั้งสองด้าน ด้านละ 1 ซม หรือเล็กว่าตามความชำนาญในการเย็บ
\item
  ขนาดอนุวาตทั้งสองด้าน ด้านละอย่างน้อย 16 ซม
  (ใหญ่เล็กกว่านี้ตามต้องการ)
\item
  ขนาดการหดเวลาเย็บกุสิและต่อผ้า ผ้าจะหดประมาณ 0.5 ซม.
  ต่อหนึ่งเส้นของการพับเย็บ และประมาณ0.7 ซม เย็บต่อหนึ่งเส้นการต่อ
\item
  ขนาดการหดเวลาซัก หรือย้อม หาได้จากการหาอัตราการหดของผ้า
  ด้วยการต้มและรีด ก่อนทำการตัด
\end{enumerate}

ในการคำนวณหาค่าของ มณฑล เราจะแยกทำทีละส่วนคือ ส่วนด้านยาว และ
ส่วนด้านสูง

\emph{ส่วนด้านยาว}

การคำนวนหาค่า มณฑล แทนค่าตามสูตรได้ดังนี้

\begin{quote}
ค่าความกว้างของมณฑลในส่วนด้านยาว ที่ได้ก็คือ 30.2 ซม. ในการตัดผ้าจริงๆ
เราอาจจะปัดเลขทศนิยมให้เป็น 30 เพื่อความสะดวกในการตัด
\end{quote}

\emph{ส่วนด้านสูง}

การคำนวนหาค่า มณฑล แทนค่าตามสูตรได้ดังนี้

\begin{quote}
ค่าความกว้างของมณฑลในส่วนด้านสูง ที่ได้ก็คือ 56 ซม. ในการตัดผ้าจริงๆ
หากมีทศนิยม เราอาจจะปัดเลขทศนิยมให้เป็นจำนวนเต็ม
\end{quote}

เมื่อได้ค่าของทั้งสองส่วนออกมาแล้วจะเขียนออกเป็นสัญญาลักษณ์ได้ว่า
ต้องการตัดจีวรขนาด 30 x 56
ก็จะเป็นที่เข้าใจในผู้ตัดเย็บจีวรของวัดบึงแสนสุขว่า ต้องวัดมณฑลด้านกว้ง
30 ซม. ด้านยาว 56 ซม.

\section{การขีดผ้าเพื่อตัด}

เมื่อได้ขนาดความยาวและสูงของ มณฑลแล้ว จะอธิบายวิธีการตัดผ้าต่อไป
ในการตัดผ้านี้มีอุปกรณ์ดังต่อไปนี้

\subsection{อุปกรณ์}

\begin{enumerate}
\def\labelenumi{(\arabic{enumi})}
\item
  พื้นที่เรียบกว้างขนาด 3 เมตรขึ้นไป ยาวอย่างน้อย 5 เมตร
  อยู่ในห้องที่ลมสงบ
\item
  ไม้ขีดอลูมินั่ม ยาว 4.5 เมตรหนึ่งอัน 2.5 เมตร หนึ่งอัน
  (สองอย่างนี้ไม่ต้องมีเส้นกำหนดขนาด) ไม้บรรทัด 60 ซม. หนึ่งอัน
\item
  ปากกาลูกลื่น สีน้ำเงิน
\item
  กรรไกรตัดผ้าชนิดดี
\item
  ตลับเมตร หรือ สายวัด
\end{enumerate}

\subsection{วิธีการขีดผ้า}

\begin{enumerate}
\def\labelenumi{(\arabic{enumi})}
\item
  ปูผ้าให้เรียบ
\item
  จัดริมผ้าด้านยาวให้ได้เส้นตรง (เท่าที่ทำได้)
  ทำได้ด้วยการขีดเส้นขอบที่ริมผ้า เพื่อตัดส่วนที่แข็งของผ้าออก
\item
  วัดหน้าผ้าด้านกว้างให้ได้ฉาก
  ทำได้ด้วยการพับผ้าให้ได้มุมฉากแล้วขีดเส้นเพื่อขลิบหัวผ้า
\item
  แทนค่าการวัดลงในแผ่นผ้า
\end{enumerate}

\subsection{วิธีการวัด และแทนค่าลงในแผ่นผ้า}

\begin{enumerate}
\def\labelenumi{(\arabic{enumi})}
\item
  หน้าผ้าด้านยาวเพื่อใช้เป็นส่วนสูงของจีวร
\item
  ส่วนสูงของจีวรจะแบ่งออกเป็น 3 ส่วน เริ่มวัดจากเส้นขลิบหัวผ้า
\end{enumerate}

\subsubsection{ส่วนที่ 1}

\begin{enumerate}
\def\labelenumi{(\arabic{enumi})}
\item
  วัดจากเส้นขลิบหัวผ้าไป 16 ซม. (ขนาดอนุวาต) เพื่อกำหนดอนุวาต จุดหมายไว้
\item
  วัดจากจุดก่อน 56 ซม. (ความกว้างมณฑล) เพื่อกำหนดความสูง อัฑฒมณฑล
\end{enumerate}

\subsubsection{ส่วนที่ 2}

\begin{enumerate}
\def\labelenumi{(\arabic{enumi})}
\item
  วัดจากจุดก่อน 6 ซม. (ความกว้างกุสิ) แล้วจุดไว้ เพื่อใช้เป็นอัฑฒกุสิ
\item
  วัดจากจุดก่อน 56ซม. เพื่อกำหนดความสูง มณฑล
\item
  วัดจากจุดก่อน 6 ซม. แล้วจุดไว้ เพื่อใช้เป็นอัฑฒกุสิ
\end{enumerate}

\subsubsection{ส่วนที่ 3}

\begin{enumerate}
\def\labelenumi{(\arabic{enumi})}
\item
  วัดจากจุดก่อน 56 ซม. เพื่อกำหนดความสูง มณฑล
\item
  วัดจากจุดก่อนไป 16 ซม. เพื่อกำหนดอนุวาต แล้วจุดไว้
\end{enumerate}

เมื่อวัดเสร็จแล้วให้ขีดเส้น เพื่อกำหนดเป็นเส้นตัด
หากพื้นที่ในการตัดยาวพอเราอาจะทำขั้นตอนทั้งหมดนี้ 2หรือ 3
ครั้งได้ในคราวเดียว

\begin{quote}
หมายเหตุ:หน้าผ้าด้านกว้างเพื่อใช้เป็นส่วนยาวของจีวร
ขนาดความกว้างของหน้าผ้าโดยทั่วไป จะวัดได้ครั้งละ 3 ขันธ์ แยกออกได้เป็น 3
ชิ้น อาจจะมีส่วนผ้าเหลือก็ให้วัดอนุวาตที่จะมาทาบขนาด 18 ซม.
ไว้เพื่อประหยัดผ้า
\end{quote}

มีวิธีการดังนี้ วัดจากริมเข้าไปประมาณ 0.5 ซม. แล้วขีด
เพื่อขลิบส่วนที่แข็งของผ้าออก และได้เส้นตรงเพื่อเป็นเส้นเริ่มในการวัด

\section{ชิ้นที่ 1 ประกอบด้วย ขันธ์ที่ 1-3}

\subsection{ขันธ์ที่ 1}

\begin{enumerate}
\def\labelenumi{(\arabic{enumi})}
\item
  วัดจากริมผ้าเข้าไป 16 ซม. จุดหมายไว้ เพื่อกำหนดอนุวาต
\item
  วัดความกว้างของมณฑล30 ซม. เริ่มจากจุดในข้อที่ผ่านมา จุดหมายไว้
\end{enumerate}

\subsection{ขันธ์ที่ 2}

\begin{enumerate}
\def\labelenumi{(\arabic{enumi})}
\item
  วัดความกว้างของกุสิ 6 ซม. เริ่มจากจุดในข้อที่ผ่านมา
\item
  วัดความกว้างของมณฑล 30 ซม. เริ่มจากจุดในข้อที่ผ่านมา จุดหมายไว้
\end{enumerate}

\subsection{ขันธ์ที่ 3}

\begin{enumerate}
\def\labelenumi{(\arabic{enumi})}
\item
  วัดความกว้างของกุสิ 6 ซม. เริ่มจากจุดในข้อที่ผ่านมา
\item
  วัดความกว้างของมณฑล 30 ซม. เริ่มจากจุดในข้อที่ผ่านมา จุดหมายไว้
\end{enumerate}

\section{ชิ้นที่ 2 ประกอบด้วยขันธ์ 4-6}

\subsection{ขันธ์ที่ 4}

\begin{enumerate}
\def\labelenumi{(\arabic{enumi})}
\item
  วัดความกว้างของกุสิ 6 ซม. เริ่มจากจุดในข้อที่ผ่านมา
\item
  วัดความกว้างของมณฑล 30 ซม. เริ่มจากจุดในข้อที่ผ่านมา จุดหมายไว้
\end{enumerate}

\subsection{ขันธ์ที่ 5}

\begin{enumerate}
\def\labelenumi{(\arabic{enumi})}
\item
  วัดความกว้างของกุสิ 6 ซม. เริ่มจากจุดในข้อที่ผ่านมา
\item
  วัดความกว้างของมณฑล 30 ซม. เริ่มจากจุดในข้อที่ผ่านมา จุดหมายไว้
\item
  วัดความกว้างของกุสิ 6 ซม. เริ่มจากจุดในข้อที่ผ่านมา
\item
  สังเกตุได้ว่า ขันธ์ที่ 5 นี้ จะมีกุสิสองอัน เพราะเป็นขันธ์กลาง
\end{enumerate}

\subsection{ขันธ์ที่ 6}

\begin{enumerate}
\def\labelenumi{(\arabic{enumi})}
\item
  วัดความกว้างของมณฑล 30ซม. เริ่มจากจุดในข้อที่ผ่านมา จุดหมายไว้
\item
  วัดความกว้างของกุสิ 6 ซม. เริ่มจากจุดในข้อที่ผ่านมา
\end{enumerate}

\section{ชิ้นที่ 3 ประกอบด้วยขันธ์ 7-9}

\subsection{ขันธ์ที่ 7}

\begin{enumerate}
\def\labelenumi{(\arabic{enumi})}
\item
  วัดความกว้างของมณฑล 30 ซม. เริ่มจากจุดในข้อที่ผ่านมา จุดหมายไว้
\item
  วัดความกว้างของกุสิ 6 ซม. เริ่มจากจุดในข้อที่ผ่านมา
\end{enumerate}

\subsection{ขันธ์ที่ 8}

\begin{enumerate}
\def\labelenumi{(\arabic{enumi})}
\item
  วัดความกว้างของมณฑล 30 ซม. เริ่มจากจุดในข้อที่ผ่านมา จุดหมายไว้
\item
  วัดความกว้างของกุสิ 6 ซม. เริ่มจากจุดในข้อที่ผ่านมา
\end{enumerate}

\subsection{ขันธ์ที่ 9}

\begin{enumerate}
\def\labelenumi{(\arabic{enumi})}
\item
  วัดความกว้างของมณฑล 30 ซม. เริ่มจากจุดในข้อที่ผ่านมา จุดหมายไว้
\item
  วัดความกว้างของอนุวาต 16 ซม. เริ่มจากจุดในข้อที่ผ่านมา
\end{enumerate}

\begin{quote}
ข้อควรสังเกตุ ขันธ์ 1 และ 9, 2 และ 8, 3 และ7, 4 และ 6
มีเส้นตำแหน่งเดียวกัน หากปะกบเข้าหากัน ส่วนขันธ์ที่ 5
จะไม่เหมือนขันธ์อื่นๆ และมีกุสิสองอัน
เคล็ดนี้สามารถนำไปใช้กับการตัดสังฆาฏิทีมีสองชั้นได้เป็นอย่างดี
\end{quote}

