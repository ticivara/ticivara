\chapter{Civara Sewing Guide}

\section{Marking the cloth}

Lay out the cloth with the \textbf{bottom side facing up}. (See how to
\href{/en/sabong-guide\#determine-the-top-and-bottom-side}{determine the
top and bottom side}.)

For a \textbf{civara}, draw the marks on the \textbf{bottom side}.

\begin{multicols}{2}
\setlength{\nextPhotoWidth}{\linewidth}

\photo{../static/img/civara/en/lay-out-the-cloth-w500.jpg}

\columnbreak

Mark the diagram lines on the cloth with a pencil. Note that the
vertical buffer length for the squaring stage is already included at the
ends of the panels. This will allow 10cm vertical manoeuvring when
joined the cloth has to be trimmed.

You can mark the panels joined end-to-end, and the vertical buffer will
be left in between.

\end{multicols}

\clearpage

\photo{../static/img/civara/en/mark-diagram-lines-w500.jpg}

\photo{../static/img/civara/en/end-to-end-w500.jpg}

Remember to leave cloth for the borders along the edges of the cloth.
The thread orientation of the border strips have to be the same as when
they are placed on the cloth, otherwise the border will shrink in a
different direction from the cloth under it. Small border pieces can be
joined if the entire length can't be obtained from the sides of the
cloth.

When the size of the cloth allow, the long borders can be cut out from
the horizontal length, and the short borders from the vertical length of
the cloth.

Cut out the panels. Take care to make clean cuts at the edges which will
be joined, if the cloth is not cut straight or the thread is allowed to
fray, it will make joining the edges difficult.

\photo{../static/img/civara/en/cut-the-cloth-w500.jpg}

\setlength{\nextPhotoWidth}{0.6\textwidth}

\photo{../static/img/civara/en/cloth-already-cut-w500.jpg}

\clearpage

\section{First and second sewing}

Make the first sewing and second sewing, same method and sewing order as
with the \href{/en/sabong-guide}{sabong}. First (1) the broken vertical
lines, (2) then the short horizontals (3) and the long continuous
vertical lines.

The finished sewing will show one thread on the outside, and two threads
on the inside.

\photo{../static/img/civara/en/first-and-second-sewing-w500.jpg}

If you iron the folds before sewing it, the crease will hold better and
the cloth needs less arranging while on the sewing machine.

\photo{../static/img/civara/en/iron-the-fold-w500.jpg}

Start sewing the broken lines from the outside of the cloth toward the
kusi.

\photo{../static/img/civara/en/first-sewing-w500.jpg}

When you reach the kusi, you can lift the needle, pull out a bit of
thread, pull the cloth to skip over the kusi, and continue sewing on the
other side.

\photo{../static/img/civara/en/skipping-the-kusi-w500.jpg}

Later, you can secure the loose threads by tying knots on the end at the
kusi. This allows to pinpoint carefully where the sewn line stops,
i.e.~how close to the line. An alternative method is to go reverse and
forward with the sewing machine at these places to lock the thread.

The thread has to be secured one way or another, because if left loose,
it can undo itself and unravel the sewing.

The outer edges of the thread (at the end of the cloth) don't have to be
tied, because it will be closed when adding the border.

\clearpage

\section{Joining: method overview}

Place the two layer on top of each other, aligned at the joining edge,
with the front sides facing each other. The front, or outer side is
where you see a single line of sewing at the kusi.

\photo{../static/img/civara/en/join-layers-on-top-w500.jpg}

\photo{../static/img/civara/en/join-layers-on-top-2-w500.jpg}

The width of the fold is determined by the distance of the first joining
thread to the edge of the upper layer, which has its back side facing
up. The lower layer extends beyond that edge.

On the lower cloth layer, the distance from the sewn line to the edge
has to be at least twice as wide or more, as the fold that you
determined on the upper layer.

If the fold is about 5mm, the complete distance should be about 10 to
15mm.

If you make the fold too narrow, there won't be space to add the second
joining thread.

Sew the two layers.

\photo{../static/img/civara/en/join-sewing-1-w500.jpg}

\photo{../static/img/civara/en/join-distance-w500.jpg}

\clearpage

Open up the two layers, and iron them toward the long overlap. The short
overlap lies on top of the long.

\photo{../static/img/civara/en/join-fold-out-w500.jpg}

\photo{../static/img/civara/en/join-fold-out-iron-w500.jpg}

While ironing, pull the cloth tight against the sewing line, so that the
ironed line doesn't add a gap where the cloth is loose.

\photo{../static/img/civara/en/join-fold-ironed-w500.jpg}

Fold the long overlap on top of the short overlap, maintaining the edge
of the cloth below. You turn the cloth over the edge, not folding the
edge along the sewing line.

\photo{../static/img/civara/en/join-fold-over-iron-w500.jpg}

\clearpage

Trim off the excess cloth.

\begin{multicols}{2}
\setlength{\nextPhotoWidth}{\linewidth}

\photo{../static/img/civara/en/join-trim-excess-w500.jpg}

\columnbreak
\setlength{\nextPhotoWidth}{\linewidth}

\photo{../static/img/civara/en/join-trim-excess-2-w500.jpg}

\end{multicols}

\vspace*{-\baselineskip}
\enlargethispage*{\baselineskip}

Fold both overlaps underneath, turn the long overlap over short one
along the edge, and crease it with the iron.

\photo{../static/img/civara/en/join-fold-both-w500.jpg}

Sew along the edge of the fold.

\photo{../static/img/civara/en/join-sew-both-w500.jpg}

The join is complete.

\photo{../static/img/civara/en/join-complete-back-w500.jpg}

\photo{../static/img/civara/en/join-complete-front-w500.jpg}

When you look at the front side, the kusi is lower than the area next to
it on the right.

If you compare it to rice paddies, the kusis are the paths between the
fields of rice. When the rice has grown and is ready to be harvested,
the field is higher than the path.

\section{Joining: marking the position}

When the first and second sewing is completed on the three pieces, lay
them out in the way they will be sewn together, to mark the cloth for
joining.

Use a long ruler to align the horizontal kusis across the three pieces.

\begin{multicols}{2}
\setlength{\nextPhotoWidth}{\linewidth}

\photo{../static/img/civara/en/lay-out-pieces-w500.jpg}

\columnbreak
\setlength{\nextPhotoWidth}{\linewidth}

\photo{../static/img/civara/en/mark-joining-points-w500.jpg}

\end{multicols}

Make a pencil mark where the kusis are going to meet. The cloth may move
or strech along the cut line, and the ends will be trimmed off at the
squaring stage, but the kusis have to stay in line.

Use pencil marks such as a V-shape, to make them recognizably different
and avoid mixing them up with the lines of other marks.

Mark both edges of the cloth so that the position is not lost later.

\photo{../static/img/civara/en/mark-kusis-w500.jpg}

Pin the two sides together where the kusis meet the other side, so that
these points don't move.

On the upper layer, two threads should be seen, on the lower layer, only
one thread.

\photo{../static/img/civara/en/pin-the-join-w500.jpg}

\photo{../static/img/civara/en/pin-the-join-close-w500.jpg}

Sewing them together will start from the kusi, toward the outer edge.

The cloth toward the edges can be allowed to move and strech a bit, the
outer edges don't have to meet exactly.

The kusis must remain in the pinned positions. If there is an extra or
missing few cm in the cloth between the two kusis, or a difference
between the upper and lower layer, the cloth has to be streched or
gathered up to absorb the difference.

\clearpage

\section{Joining: first sewing}

Start sewing from one end of the kusi, across the kusi and moving
outward.

When starting from one side, the kusi shows the sewing lines:

\photo{../static/img/civara/en/sewing-from-kusi-w500.jpg}

When starting from another side, the sewing lines are not visible:

\photo{../static/img/civara/en/sewing-from-kusi-2-w500.jpg}

\photo{../static/img/civara/en/sewing-from-kusi-3-w500.jpg}

\photo{../static/img/civara/en/sewing-from-kusi-4-w500.jpg}

When you have sewn to the end, turn the cloth around, start from the
kusi again in the other direction.

(If you start sewing from the outside in, the kusis will not necessary
meet.)

\clearpage

\section{Joining: folding and second sewing}

The following photos show the above joining method on the civara cloth.

\photo{../static/img/civara/en/join-iron-civara-w500.jpg}

\setlength{\nextPhotoWidth}{0.5\textwidth}

\photo{../static/img/civara/en/join-iron-civara-2-w500.jpg}

\photo{../static/img/civara/en/join-iron-civara-3-w500.jpg}

\photo{../static/img/civara/en/join-iron-civara-6-w500.jpg}

\photo{../static/img/civara/en/join-iron-civara-6a-w500.jpg}

\photo{../static/img/civara/en/join-iron-civara-4-w500.jpg}

\photo{../static/img/civara/en/join-iron-civara-5-w500.jpg}

\photo{../static/img/civara/en/join-civara-trim-w500.jpg}

\photo{../static/img/civara/en/join-civara-trim-2-w500.jpg}

\photo{../static/img/civara/en/join-civara-fold-over-w500.jpg}

\photo{../static/img/civara/en/join-civara-fold-iron-w500.jpg}

\photo{../static/img/civara/en/join-civara-fold-sewing-w500.jpg}

\clearpage

\section{Squaring}

Lay out the joined cloth, top side facing down. You should see the two
threads of the back, or inner side.

Tape the cloth to the floor. You only want it to lie flat, not streched.
If the shape of the cloth is distorted, the trimming lines will be wavy.
Attach the tape at the sewn lines, where the cloth doesn't strech as
much.

Use the short side as the base and align the rulers at 90 degrees.

Find the line which you can cut straight along the edge without gaps.

\begin{multicols}{2}
\setlength{\nextPhotoWidth}{\linewidth}

\photo{../static/img/civara/en/squaring-civara-lay-out-w500.jpg}

\columnbreak
\setlength{\nextPhotoWidth}{\linewidth}

\photo{../static/img/civara/en/squaring-civara-rulers-w500.jpg}

\end{multicols}

Mark the line with a pencil.

Later, when you put the border on the cloth, in the border area the
cloth is doubled and less prone to streching than the middle part of the
robe. The result is that over time, the middle part of the robe sags and
doesn't fold into a neat line.

This can be counteracted by shortening the cloth at the middle.

Find the horizontal center point of the robe and mark a point 1.5cm
above the line at the edge. Mark a line from this point to the corners,
creating a roof or \texttt{Λ} shape. Do this at both the top and bottom
side.

\photo{../static/img/civara/en/squaring-civara-midpoint-w500.jpg}

Cut the cloth along the marked lines.

\photo{../static/img/civara/en/squaring-cut-w500.jpg}

\section{Borders}

Add the \href{/en/borders}{borders}.

