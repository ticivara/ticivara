\chapter{Sabong Sewing Guide}

\section{Planning the cutting layout}

Measure the cloth and plan how the sabong and the borders are going to
fit.

If the cloth is not wide enough for the sabong to be drawn and cut as
one piece, determine which khandas will be cut out separately. Draw them
horizontally on the cloth, instead of the vertical orientation seen on
the pattern diagram.

On the drawing, the cutting lines are the long vertical lines.

In the case of separate khandas, the pattern calculator implies a 1cm
cutting buffer at the side edges where khandhas will be joined, and
5-10cm shifting buffers at the top and bottom edge of the khandhas.

The separate khandas will be joined along these lines after the first
and second sewing is completed.

The thread orientation of the cloth has to be observed and kept
consistent between the separate pieces. The cloth streches to a
different degree when pulled in the vertical or horizontal direction. If
this is mixed up, the different orientation will cause the material to
wrinkle.

The borders may fit in one length, but may have to be sewn from joining
separate pieces to maintain the thread orientation.

First, determine how much of the total cloth is going to be needed to
fit the sabong and the borders, and cut off this appriximate size from
the main cloth roll.

Wash it at 40 degrees, to make sure the cloth shrinks before drawing the
marks.

Dry and iron.

\section{Marking the cloth}

Determine the final size with the pattern calculator and mark the cloth.

Keep in mind the top and bottom side of the cloth. The top side is going
to be slightly more reflective when looking at it from a low angle.

\section{First sewing}

Start with the broken lines. Pinch the cloth and crease it along the
drawn line.

Fold and make the first sewing, at about 5mm from the edge.

Keep in mind to not sew too narrow, the second sewing will be made
between this sewn line and the edge of the fold.

Observe the sequence of the lines: (1) broken vertical, (2) short
horizontal, (3) long vertical.

\photo{../static/img/sabong/photos/first-sewing-broken-lines-w500.jpg}

\photo{../static/img/sabong/photos/first-sewing-horiz-lines-w500.jpg}

The beginning and end of the lines don't need to be locked with reverse
sewing. Around the edge of the cloth, the hem and the boder will lock
the threads. At the kusis where the broken lines end, the next line will
cross it and lock the thread.

When sewing the broken lines, one may start at one end of the cloth,
stop where the line crosses the kusi. The needle can be then lifted
across the kusi, continuing to sew the broken line on the other side of
the kusi.

After one group of lines is finished, such as after the broken lines,
pinch and crease the cloth along one of the lines in the next group.
Fold and sew along the edge, closing the T at the place where the lines
meet.

\section{Second sewing}

The second sewing is in the same sequence as the first.

For the second sewing, fold the cloth outward from the kusis, flatten
and sew between the first sewing and the edge of the fold.

\photo{../static/img/sabong/photos/second-sewing-broken-lines-1-w500.jpg}

\photo{../static/img/sabong/photos/second-sewing-broken-lines-2-w500.jpg}

This way on the outside of the cloth there will be one visible sewing
line, and on the inside there will be two.

In the second sewing, the folding direction has to be observed so that
the kusis are lower than the middle sections.

If it was cut from separate pieces, these have to be joined now. The
join has to hide both open edges. The joining fold is made by two sewing
lines, which are effectively the first and second sewing along the long
vertical lines.

\section{Trim and Hem}

When sewing from separate pieces, they may need to be trimmed along the
edges all around the cloth, to create even and straight edges.

Make a hem by folding up 1cm around the edges and sewing them.

\section{Borders}

Add the \href{/en/borders}{borders}.

