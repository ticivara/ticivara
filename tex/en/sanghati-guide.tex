\chapter{Sanghati Sewing Guide}

\section{Drawing and Cutting}

Cut off the length of the cloth and iron it, so that the cloth lies
smooth and flat.

Cut off 10cm longer on each side to allow for mistakes.

Fold up a short length, align the edges to be parallel, this will
establish a 90 degree corner. Crease the corners of the fold to create a
mark.

\photo{../static/img/sanghati/figures/fold-up-for-90deg.jpg}

Fold out and use the creased marks to measure where the edge of the
drawing is going to start.

Mark the bottom edge above the holes of the salvage on the edge. Use
this as the bottom side of the drawing.

Identify the top and bottom side of the cloth. The top side is slightly
more reflective, and the bottom is more matt. When a corner is folded
over, the difference should be possible to tell when looking at the
cloth at a low angle.

\photo{../static/img/sanghati/figures/fold-over-for-top-bottom.jpg}

Draw the marks on the top side.

Mark the lenghts and draw the pattern.

On the \emph{khanda lines} that you are intending to cut, carry the
crossing line over a few centimeters to keep a mark of the positions
after the cut.

\photo{../static/img/sanghati/figures/mark-and-cut.jpg}

After the lines are drawn, cut the edges around, and cut the dividing
line of the khandas.

For the second layer, the already marked cloths can be placed over the
blank cloth and the marks can be copied over easily.

The top and bottom side of the cloth have to be observed. The marks
again have to be drawn on the top side of the second layer, but the
first layer have to be flipped on the vertical axis when marking, to
produce a mirror image.

\photo{../static/img/sanghati/figures/copy-the-marks.jpg}

\section{Cutting the Borders}

See \href{/en/borders}{the Border Sewing Guide}

\section{First Sewing}

Start the first sewing from the broken lines (1).

\photo{../static/img/sanghati/figures/first-sewing-sequence.jpg}

\photo{../static/img/sanghati/figures/first-sewing-fold.jpg}

Sew on the marked side. Start from a hand-span away from the kusi, lock
the thread at the start with reverse sewing.

Sew towards the kusi, and stop at 2mm before the kusi line.

Move to the other side of the kusi, and continue the line. Lock the far
edges by reverse sewing. The inner edge doesn't have to be locked, the
second sewing will cross and lock it.

\photo{../static/img/sanghati/photos/first-sewing-w500.jpg}

\section{Joining the Layers}

Start with the middle khandha, take both layers and arrange them with
the sewing inside.

Take the horizontal kusi line and fold it down on the two sides.

\photo{../static/img/sanghati/figures/join-layers.jpg}

Pin them below the sewn line and fold back to check that the kusi will
be on a lower level than the middle section.

\photo{../static/img/sanghati/photos/join-horizontal-pinned-w500.jpg}

\photo{../static/img/sanghati/photos/join-horizontal-fold-out-w500.jpg}

This is also called the ``canal'' and the ``shore'', if you imagine the
kusis being the watering canals between rice paddy fields.

Sew the layer together below the first sewing, within 1-2 mm. Sew both
horizontal kusi lines this way.

\photo{../static/img/sanghati/figures/sew-horizontal-lines.jpg}

\photo{../static/img/sanghati/photos/join-sew-below-line-w500.jpg}

It is possible to fold the first one in the wrong direction, and when
you fold it back, the kusi ends up higher. When folded in the wrong
direction, there will be a flap at the corner.

When it is folded in the right direction, the flap will be inside.

\photo{../static/img/sanghati/photos/fold-before-join-w500.jpg}

\photo{../static/img/sanghati/photos/fold-flap-inside-w500.jpg}

Pin the layers, and double-check by folding it back out, that the kusi
is lower than the middle section.

Join the horizontal lines this way.

\photo{../static/img/sanghati/photos/fold-joined-w500.jpg}

Join the vertical lines in the same manner, but stop at 2mm from the
horizontal lines.

\photo{../static/img/sanghati/figures/sew-vertical-lines.jpg}

\photo{../static/img/sanghati/photos/join-sew-close-w500.jpg}

\section{Joining the Khandhas}

Join one of the side khandhas to the middle khandha.

\photo{../static/img/sanghati/figures/join-khandhas.jpg}

Measure the visible kusi width on the middle khandha. If you started
with a 6cm kusi, and did the sewing at 2mm on both lines, the result
shuld be about 5.5mm visible kusi width on the cloth.

When joining the side khandha, allow 1cm from the edge for sewing.

To get a 5.5cm kusi, mark the cloth at 6.5cm and trim the egde.

\photo{../static/img/sanghati/figures/trim-the-edge.jpg}

Pull the sewing line against the edge of a ruler to straighten the
cloth.

\photo{../static/img/sanghati/photos/khandhas-prepare-to-trim-w500.jpg}

Trim.

\photo{../static/img/sanghati/photos/khandhas-trim-before-join-w500.jpg}

Place the side layers on top and bottom of the middle layer. Find the
pencil marks of the kusi lines and align them evenly.

\photo{../static/img/sanghati/figures/join-sides.jpg}

\photo{../static/img/sanghati/photos/khandhas-align-kusis-pencil-mark-w500.jpg}

\photo{../static/img/sanghati/photos/khandhas-align-kusis-pencil-mark-closeup-w500.jpg}

\photo{../static/img/sanghati/photos/khandhas-align-kusis-w500.jpg}

Pin the four layers and sew them at 1cm from the edge of the cloth.
Before sewing, you can double-check the arrangement by folding the
layers out along the pins. Check that the kusi lower, and kusis are
aligned across the cloth.

\photo{../static/img/sanghati/photos/khandhas-pin-edges-w500.jpg}

\photo{../static/img/sanghati/photos/khandhas-pin-edges-closeup-w500.jpg}

\photo{../static/img/sanghati/photos/khandhas-fold-out-w500.jpg}

\photo{../static/img/sanghati/photos/khandhas-before-sewing-w500.jpg}

\photo{../static/img/sanghati/photos/khandhas-join-sewing-w500.jpg}

After sewing, trim the edge at 3mm from the sewn line.

\photo{../static/img/sanghati/photos/khandhas-trim-after-join-w500.jpg}

The other lines of the side khandhas are joined the same way as before.

Continue with joining the other side to the middle khandha and proceed
outward until all the khandhas are joined.

\section{Sewing the Borders}

See \href{/en/borders}{Borders}.

