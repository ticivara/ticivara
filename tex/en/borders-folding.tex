\chapter{Borders: Folding Medthod}

This is an alternative method to add the borders, by placing them on the
outside first, then folding them over to reverse them out. This method
is sometimes called \emph{French corners}.

Prepare the border strips in the same way as in the
\href{/en/borders}{first border guide}.

Check that the border strips extend beyond the main cloth at least by
about 5 cm.

After the second sewing is complete on the main cloth, place the border
on the \emph{outside} of the main cloth, with the folded hem of the
border on the inner edge.

The outside is the one with a single sewing line visible.

\photo{../static/img/borders/photos/border-folding-place-on-the-outside-w500.jpg}

Start the sewing at about 10-15 cm from the corner, to allow working on
the corner later.

\clearpage

Sew the edge while keeping the main cloth aligned with the border strip.

\begin{multicols}{2}
\setlength{\nextPhotoWidth}{\linewidth}

\photo{../static/img/borders/photos/borders-folding-sew-the-edge-1-w500.jpg}

\columnbreak
\setlength{\nextPhotoWidth}{\linewidth}

\photo{../static/img/borders/photos/borders-folding-sew-the-edge-2-w500.jpg}

\end{multicols}

\photo{../static/img/borders/photos/borders-folding-sew-the-edge-3-w500.jpg}

Continue with sewing the other border strips on the outer edges.
Remember to leave space at the corners.

\photo{../static/img/borders/photos/borders-folding-more-edges-w500.jpg}

After attaching all the border strips, lay out the cloth.

\begin{multicols}{2}
\setlength{\nextPhotoWidth}{\linewidth}

\photo{../static/img/borders/photos/borders-folding-layout-cloth-w500.jpg}

\columnbreak
\setlength{\nextPhotoWidth}{\linewidth}

\photo{../static/img/borders/photos/borders-folding-corner-details-1-w500.jpg}

\end{multicols}

Align the layers at the corner so that the edges are flush, and mark the
diagonal on the top and bottom border layers.

\setlength{\nextPhotoWidth}{0.6\textwidth}

\photo{../static/img/borders/photos/borders-folding-corner-details-2-w500.jpg}

\photo{../static/img/borders/photos/borders-folding-corner-details-3-w500.jpg}

\clearpage

To help finding the position later, make a pencil mark in inner the
corner on the main cloth.

\photo{../static/img/borders/photos/borders-folding-corner-details-4-w500.jpg}

\photo{../static/img/borders/photos/borders-folding-corner-details-5-w500.jpg}

\clearpage
\enlargethispage*{\baselineskip}

Cut the border strips at a distance on the outer side of the diagonal.

\photo{../static/img/borders/photos/borders-folding-corner-details-6-w500.jpg}

\photo{../static/img/borders/photos/borders-folding-corner-details-7-w500.jpg}

\photo{../static/img/borders/photos/borders-folding-corner-details-8-w500.jpg}

Align and pin the corner layers, then sew along the marked line to join.

\photo{../static/img/borders/photos/borders-folding-corner-join-1-w500.jpg}

\setlength{\nextPhotoWidth}{0.6\textwidth}
\enlargethispage*{\baselineskip}

\photo{../static/img/borders/photos/borders-folding-corner-join-2-w500.jpg}

Iron the joined line to flatten the cloth.

\photo{../static/img/borders/photos/borders-folding-corner-join-3-w500.jpg}

Sew along the edge, completing the line on the outer edge of the cloth.

\photo{../static/img/borders/photos/borders-folding-corner-join-4-w500.jpg}

Trim the very corners at a low angle, to reduce the cloth material which
will be inside the corner when reversed out.

\photo{../static/img/borders/photos/borders-folding-corner-join-5-w500.jpg}

\clearpage

Fold over to reverse out the corners. Use a blunt point, for example the
tip of a pencil, to arrange the inside of the tip of the corner.

\photo{../static/img/borders/photos/borders-folding-corner-reversed-w500.jpg}

Open the inner side of the border and iron the edge to make the edge
sharply creased.

\photo{../static/img/borders/photos/borders-folding-corner-flattened-w500.jpg}

\clearpage

Lay out the cloth and pin the loose inner edges.

\photo{../static/img/borders/photos/borders-folding-layout-pin-w500.jpg}

Add the final sewing lines along the inner and outer edge of the border.

\photo{../static/img/borders/photos/borders-folding-final-lines-w500.jpg}

