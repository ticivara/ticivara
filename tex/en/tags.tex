\chapter{Tag Knots}

On the \emph{civara} and \emph{sanghati}, the tags function as buttons
to fasten the loose ends of the robe. These are made by tying small
strings into a knot and a loop. The button side is otherwise known as
the \emph{Chinese Button Knot}.

The string used below is thicker than usual, to help seeing the
structure better.

Start with the string around a pencil or a finger.

\photo{../static/img/knot/knot_01_w500.jpg}

Bring the right strand across in an over-hand loop.

\photo{../static/img/knot/knot_02_w500.jpg}

\clearpage

Take the loop, and give it a half-twist, away from you, turning the
lower edge over to the top.

\photo{../static/img/knot/knot_03_w500.jpg}

Pull the left strand through the loop.

\photo{../static/img/knot/knot_04_w500.jpg}

The left strand goes under the right,

\photo{../static/img/knot/knot_05_w500.jpg}

\clearpage

and through the new loop, as you see below.

\photo{../static/img/knot/knot_06_w500.jpg}

Pull in the slack, but don't tighten the knot.

\photo{../static/img/knot/knot_07_w500.jpg}

Pulling further\ldots{}

\photo{../static/img/knot/knot_08_w500.jpg}

\clearpage

The knot should have this form, in a figure-of-eight and a diamond shape
in the center.

\photo{../static/img/knot/knot_09_w500.jpg}

Take the right strand under and through the diamond shape in the center.

\photo{../static/img/knot/knot_10_w500.jpg}

Pull in the slack.

\photo{../static/img/knot/knot_11_w500.jpg}

\clearpage

Same with the left strand, take it under and through the center.

\photo{../static/img/knot/knot_12_w500.jpg}

The structure of the knot is complete.

Pull in the slack, remove the pencil and keep tightening the knot while
maintaining its form.

\photo{../static/img/knot/knot_13_w500.jpg}

\photo{../static/img/knot/knot_14_w500.jpg}

\photo{../static/img/knot/knot_15_w500.jpg}

\photo{../static/img/knot/knot_16_w500.jpg}

\photo{../static/img/knot/knot_17_w500.jpg}

